\chapter{Introduction}

The use of radioactive isotopes for medical purposes has been investigated
since 1920, and since 1940 attempts have been undertaken at imaging
radionuclide concentration in the human body.

In the early 1950s, Ben Cassen introduced the rectilinear scanner, a
``zero-dimensional'' scanner, which (very) slowly scanned in two
dimensions to produce a two-dimensional image of the radionuclide
concentration in the body. In the late 1950s, Hal Anger developed the
first ``true'' gamma camera, introducing an approach that is still
being used in the design of virtually all modern camera's: the Anger
scintillation camera, a 2D planar detector to produce a 2D projection
image without scanning.

The Anger camera can also be used for tomography. The projection images can
then be used to compute the original spatial distribution of the radionuclide
within a slice or a volume. Already in 1917, Radon published the mathematical
method for reconstruction from projections, but only in the 1970s, the method
was introduced in medical applications, first in CT and next in nuclear
medicine imaging. At the same time, iterative reconstruction methods were
being investigated, but the application of those methods had to wait for
sufficient computer power till the 1980s.

The Anger camera is often called gamma camera, because it detects gamma rays.
When it is designed for tomography, it is also called a SPECT camera. SPECT
stands for Single Photon Emission Computed Tomography and contrasts with PET,
i.e.\ Positron Emission Tomography, which detects photon pairs. Anger showed
that two scintillation camera's could be combined to detect photon pairs
originating after positron emission. Ter-Pogossian et al.\ built the first
dedicated PET-system in the 1970s, which was used for phantom studies.  Soon
afterwards, Phelps, Hoffman et al built the first PET-scanner (also called
PET-camera) for human studies. Since its development,
the PET-camera has been regarded nearly exclusively as a research system.
Only in about 1995, it became a true clinical instrument.

Below, PET and SPECT will be discussed together since they have a lot in
common. However, there are also important differences. One is the cost price:
PET systems are about 4 times as expensive as gamma cameras. In addition, many
PET-tracers have a very short half life (i.e. the time after which the
radioactivity decreases to 50\%), so it is mandatory to have a small
cyclotron, a laboratory and a radiopharmacy expert in the close neighborhood
of the PET-center.

An excellent book on this subject is ``Physics in Nuclear Medicine'',
by Cherry, Sorenson and Phelps \cite{Cherry}. Two useful books have
been published in the Netherlands, one to provide insight
\cite{Aanbevelingen}, the other describing procedures \cite{Leerboek}.
An excellent book for researchers in the field is the one of H.\
Barrett and K.\ Meyers \cite{Barrett}. The International Atomic Energy
Agency (IAEA) publishes books with free online access, e.g. \cite{IAEA}.
%Springer has interesting free books on the internet as well, including
%;\cite{Cantone} - \cite{Zaidi}.

